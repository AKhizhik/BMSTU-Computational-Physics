% !TEX encoding = UTF-8 Unicode
\documentclass[
11pt,
master, % тип документа
subf, % подключить и настроить пакет subfig для вложенной нумерации рисунков
href, % подключить и настроить пакет hyperref
colorlinks=true, % цветные гиперссылки
times, % шрифт Times как основной
%fixint=false % отключить прямые знаки интегралов
]{disser}
\usepackage[left=25mm, top=20mm, right=10mm, bottom=20mm]{geometry}
\usepackage[T2A]{fontenc}
\usepackage[utf8]{inputenc}
\usepackage[english,russian]{babel}
\usepackage{amsmath,amssymb,cmap} % cmap для кодировки шрифтов в pdf
\usepackage{pdfpages} % вставляем pdf файлы
\usepackage{indentfirst} % отделять первую строку раздела абзацным отступом
\usepackage{titletoc} % убираем отступ перед "Оглавление"
\usepackage{graphicx}
\usepackage{setspace}
\usepackage{verbatim} % для оформления кода
\usepackage{pdfsync} % установка соответствия документ - код
\graphicspath{{./Img/}}

\setlength\parindent{5ex} % абзацный отступ равный пяти строчным буквам основного шрифта
\pagestyle{plain} % включаем нумерацию
\setcounter{tocdepth}{2} % включать подсекции в оглавление
\linespread{1.3} % полуторный интервал


% Номера страниц снизу и по центру
\pagestyle{footcenter}
\chapterpagestyle{footcenter}

\begin{document}
	
	\pagestyle{empty}
	\begin{center}
		
		\noindent  Федеральное государственное бюджетное образовательное учреждение\\
		высшего профессионального образования\\
		
		Московский государственный технический университет им. Н.Э. Баумана \\
		Факультет <<Фундаментальные науки>>\bigskip\\
		
		\vfill
		
		Домашнее задание\\
		по курсу «Вычислительная физика»\\
		на тему: «Решение интегральных уравнений»\\
		
		
		\vfill
		\vfill
		\begin{flushright}
			\begin{tabular}{ll}
				Выполнили: & студенты группы ФН4-72Б     \\
				& Хижик А.И., Мистрюкова Л.А.,  \\
				Проверил:  & доцент, к.физ.-мат.н.       \\
				& Хасаншин Р.Х.
			\end{tabular}
		\end{flushright}
		\vfill
		\begin{center}
			Москва, $2019$
		\end{center}
		
	\end{center}
	\pagebreak
	
	
	\pagestyle{plain}
	\tableofcontents

\section{Интегральные уравнения Фредгольма}
Решите неоднородные уравнения Фредгольма второго рода методом последовательных приближений (найти $y_4(x)$ и оценить погрешность), методом вырожденного ядра и с использованием резольвенты.

\subsection{$\displaystyle y(x) = \lambda \int_{0}^{1} x s y(s)ds + x$}

$\displaystyle K(x,s) = K_1(x,s) = xs$, $f(x) = \varphi_0(x) = x$

\subsubsection{Метод последовательных приближений}

\fbox{
 \addtolength{\linewidth}{-40\fboxsep}
 \addtolength{\linewidth}{-40\fboxrule}
 \begin{minipage}{\linewidth}
  \begin{equation}
    \varphi_0(x) = f(x), \;\; \varphi_m(x) = \int_{a}^{b} K(x,s) \varphi_{m-1}(s) ds = \int_{a}^{b} K_m(x,s) f(s)ds
  \end{equation}
    \begin{equation}
    K_m(x,s) = \int_{a}^{b} K(x,t)K_{m-1}(t,s)dt
  \end{equation}
  \begin{equation}
    y(x) = \sum_{m=0}^{\infty} \lambda^m \varphi_m(x)
  \end{equation}
 \end{minipage}
}

\begin{equation}
 \left.\begin{aligned}
        K_2 = \int_{0}^{1} x t^2 s dt &= \frac{1}{3} xs\\
        K_3 = \int_{0}^{1} \frac{1}{3} x t^2 s dt &= \frac{1}{9} xs
       \end{aligned}
 \right\}
 \qquad \longrightarrow K_m = \frac{1}{3^{m-1}} xs \Rightarrow \varphi_m(x) = \frac{1}{3^m}x
\end{equation}

$$y(x) = x \sum_{m=0}^{\infty} \left(\frac{\lambda}{3}\right)^m = \frac{3x}{3-\lambda}$$

$$y_4(x) = \frac{\left(3 - \frac{\lambda^5}{81}\right)x}{3 - \lambda}$$

$$\Delta = \frac{\lambda^5 x}{243-81 \lambda}$$

\subsubsection{Резольвента}

\fbox{
 \addtolength{\linewidth}{-80\fboxsep}
 \addtolength{\linewidth}{-80\fboxrule}
 \begin{minipage}{\linewidth}
  \begin{equation}
    R(x,s,\lambda) = \sum_{m=1}^{\infty} \lambda^{m-1} K_m(x,s)
  \end{equation}
  \begin{equation}
    y(x) = \lambda \int_{a}^{b} R(x,s,\lambda) f(s) ds + f(x)
  \end{equation}
 \end{minipage}
}


$$R(x,s,\lambda) = xs \sum_{m=1}^{\infty} \left(\frac{\lambda}{3}\right)^{m-1} = \frac{3 xs}{3 - \lambda} $$
$$y(x) = \frac{3x}{3 - \lambda}$$

\subsubsection{Метод вырожденного ядра}

\fbox{
 \addtolength{\linewidth}{-30\fboxsep}
 \addtolength{\linewidth}{-30\fboxrule}
 \begin{minipage}{\linewidth}
  \begin{equation}
    K(x,s) = \sum_{i=1}^{n} \alpha_i(x) \beta_i(s)
  \end{equation}
  \begin{equation}
    c_i - \lambda \sum_{j=1}^{n} a_{ij} c_j = f_i, \;\; a_{ij} = \int_{a}^{b} \alpha_i(s) \beta_j(s) ds, \;\; f_i = \int_{a}^{b} f(s) \beta_i(s) ds, \;\; i = \overline{1,n}
  \end{equation}
  \begin{equation}
    y(x) = \lambda \sum_{i=1}^{n} c_i \alpha_i(x) + f(x)
  \end{equation}
 \end{minipage}
}

$$\alpha(x) = x,\;\beta(s) = s \rightarrow a_{11} = \int_{0}^{1} s^2 ds = \frac{1}{3}, \; f_1 = \int_{0}^{1} s^2 ds = \frac{1}{3} \Rightarrow c_1 - \lambda a_{11} c_1 = f_1 \Longleftrightarrow c_1 = \frac{1}{3 - \lambda}$$

$$y(x) = \frac{3x}{3 - \lambda}$$

\subsection{$\displaystyle y(x) = \lambda \int_{0}^{1}  y(s)ds +  \mathrm{sin} (\pi x)$}

$K(x,s) = K_1(x,s) = 1$, $f(x)= \varphi_0(x) = \sin(\pi x)$

\subsubsection{Метод последовательных приближений}

\begin{equation}
 \left.\begin{aligned}
        K_2 &= 1\\
        K_3 &= 1
       \end{aligned}
 \right\}
 \qquad \longrightarrow K_m = 1 \Rightarrow \varphi_m(x) = \frac{2}{\pi}
\end{equation}

$$y(x) = \sin(\pi x) + \frac{2}{\pi} \sum_{m=1}^{\infty} \lambda^m = \sin(\pi x) + \frac{2 \lambda}{\pi(1 - \lambda)}$$

$$y_4(x) = \frac{2 \lambda  \left(\lambda^4-1\right)}{\pi  (\lambda -1)}+\sin(\pi  x)$$

$$\Delta = \frac{2\lambda ^5}{\pi(1 -\lambda)}$$

\subsubsection{Резольвента}

$$R(x,s,\lambda) = \sum_{m=1}^{\infty} \lambda^{m-1} = \frac{1}{1 - \lambda}$$

$$y(x) = \frac{2\lambda}{\pi(1 - \lambda)} + \sin(\pi x)$$

\subsubsection{Метод вырожденного ядра}

$$\alpha(x) = 1,\;\beta(s) = 1 \rightarrow a_{11} = \int_{0}^{1} ds = 1, \; f_1 = \int_{0}^{1} \sin(\pi x) ds = \frac{2}{\pi} \Rightarrow c_1 - \lambda a_{11} c_1 = f_1 \Longleftrightarrow c_1 = \frac{2}{\pi(1-\lambda)}$$

$$y(x) = \frac{2\lambda}{\pi(1 - \lambda)} + \sin(\pi x)$$

\subsection{$\displaystyle y(x) = \lambda \int_{0}^{1} \frac{x}{s^2 + 1} y(s)ds + x^2 + 1$}

$K(x,s) = K_1(x,s) = \frac{x}{s^2 + 1}$, $f(x)= \varphi_0(x) = x^2 + 1$

\subsubsection{Метод последовательных приближений}

\begin{equation}
 \left.\begin{aligned}
        K_2 = \int_{0}^{1} \frac{xt}{(t^2 + 1)(s^2 + 1)} &= \frac{\log(2)}{2} \frac{x}{s^2 + 1}\\
        K_3 = \int_{0}^{1} \frac{\log(2)}{2} \frac{xt}{(t^2 + 1)(s^2 + 1)} &= \left(\frac{\log(2)}{2}\right)^2 \frac{x}{s^2 + 1}
       \end{aligned}
 \right\}
 \qquad \longrightarrow K_m = \left(\frac{\log(2)}{2}\right)^{m-1} \frac{x}{s^2 + 1}
\end{equation}

$\displaystyle \Rightarrow \varphi_m(x) = \left(\frac{\log(2)}{2}\right)^{m-1} x$

$$y(x) = x^2 + 1 + x\sum_{m=1}^{\infty} \lambda^m \left(\frac{\log(2)}{2}\right)^{m-1} = x^2 + \frac{2 \lambda x}{2 - \lambda \log(2)} + 1$$

$$y_4(x) = x^2+\frac{2 \lambda  x \left(\frac{1}{16} \lambda ^4 \log ^4(2)-1\right)}{\lambda  \log(2)-2}+1$$

$$\Delta = \frac{\lambda ^5 x \log ^4(2)}{8(2 - \lambda  \log (2))}$$

\subsubsection{Резольвента}

$$R(x,s,\lambda) = \sum_{m=1}^{\infty} \left(\frac{\lambda \log(2)}{2}\right)^{m-1} \frac{x}{s^2 + 1}$$

$$y(x) = \lambda x \sum_{m=1}^{\infty} \left(\frac{\lambda \log(2)}{2}\right)^{m-1} + x^2 + 1 = x^2 + \frac{2 \lambda x}{2 - \lambda \log(2)} + 1$$

\subsubsection{Метод вырожденного ядра}

$$\alpha(x) = x,\;\beta(s) = \frac{1}{s^2+1} \rightarrow a_{11} = \int_{0}^{1} \frac{s}{s^2+1} ds = \frac{\log(2)}{2}, \; f_1 = \int_{0}^{1} ds = 1 \Rightarrow$$
$$\Rightarrow c_1 - \lambda a_{11} c_1 = f_1 \Longleftrightarrow c_1 = \frac{2}{2 - \lambda \log(2)}$$

$$y(x) = \frac{2\lambda x}{2 - \lambda \log(2)} + x^2 + 1$$

\subsection{$\displaystyle y(x) = \lambda \int_{0}^{\pi} x~ \mathrm{sin} (2s) y(s)ds + \mathrm{cos}(2x)$}

$K(x,s) = K_1(x,s) = x \sin(2s)$, $f(x)= \varphi_0(x) = \cos(2x)$

\subsubsection{Метод последовательных приближений}

\begin{equation}
 \left.\begin{aligned}
        K_2 = \int_{0}^{\pi} x t \sin(2t) \sin(2s) dt &= - \frac{\pi}{2} x\sin(2s)\\
        K_3 = \int_{0}^{\pi} \left(-\frac{\pi}{2}\right) x t \sin(2t) \sin(2s) dt &= \left(\frac{\pi}{2}\right)^2 x\sin(2s)
       \end{aligned}
 \right\}
 \qquad \longrightarrow K_m = \left(-\frac{\pi}{2}\right)^{m-1} x\sin(2s)  \Rightarrow \varphi_m(x) = 0
\end{equation}

$$y(x) = \cos(2x)$$

$$y_4(x) = \cos(2x)$$

$$\Delta = 0$$

\subsubsection{Резольвента}

$$R(x,s,\lambda) = \sum_{m=1}^{\infty} \left(-\frac{\pi \lambda}{2}\right)^{m-1} x \sin(2s)$$

$$y(x) = \cos(2x)$$

\subsubsection{Метод вырожденного ядра}

$$\alpha(x) = x,\;\beta(s) = \sin(2s) \rightarrow a_{11} = \int_{0}^{\pi} s \sin(2s) ds = -\frac{\pi}{2}, \; f_1 = \int_{0}^{1} \cos(2s) \sin(2s) ds = 0 \Rightarrow$$
$$\Rightarrow c_1 - \lambda a_{11} c_1 = f_1 \Longleftrightarrow c_1 = 0$$

$$y(x) = \cos(2x)$$

\subsection{$\displaystyle y(x) = \lambda \int_{0}^{2\pi} \mathrm{sin} (x+s) y(s)ds + 2$}

$K(x,s) = K_1(x,s) = \sin(x+s)$, $f(x)= \varphi_0(x) = 2$

\subsubsection{Метод последовательных приближений}

\begin{equation}
 \left.\begin{aligned}
        K_2 = \int_{0}^{2\pi} \sin(x+t) \sin(t+s) dt &= \pi \cos(s-x)\\
        K_3 = \int_{0}^{2\pi} \pi \sin(x+t) \cos(s-t) dt &= \pi^2 \sin(s+x)
       \end{aligned}
 \right\}
 \qquad \varphi_2(x) =0, \;\; \varphi_3(x) = 0
\end{equation}

$$y(x) = 2$$

$$y_4(x) = 2$$

$$\Delta = 0$$


\subsubsection{Резольвента}

$$R(x,s,\lambda) = 0$$

$$y(x) = 2$$

\subsubsection{Метод вырожденного ядра}

$\displaystyle K(x,s) = \sin(x+s) = \cos(x) \sin(s) + \cos(s) \sin(x)$

$\displaystyle\alpha_1(x) = \cos(x),\;\;\beta_1(s) = \sin(s),\;\;\alpha_2(x) = \sin(x),\;\;\beta_2(s) = \cos(s)  \rightarrow a_{11} = a_{22} = 0,\\
a_{12} = a_{21} = \pi,\;\; f_1 = f_2 = 0 \Rightarrow c_i - \lambda a_{ij} c_j = f_i,\;\; i,j = 1,2 \Longleftrightarrow c_1 = c_2 = 0$

$$y(x) = 2$$

\subsection{$\displaystyle y(x) = \int_{0}^{\frac{\pi}{2}} \mathrm{sin} (x) \mathrm{cos}(s) y(s)ds + 1$}

$K(x,s) = K_1(x,s) = \sin(x)\cos(s)$, $f(x)= \varphi_0(x) = 1$

\subsubsection{Метод последовательных приближений}

\begin{equation}
 \left.\begin{aligned}
        K_2 = \int_{0}^{\frac{\pi}{2}} \sin(x)\cos(t)\sin(t)\cos(s)dt &= \frac{1}{2} \sin(x) \cos(s)\\
        K_3 = \int_{0}^{\frac{\pi}{2}} \frac{1}{2}\sin(x)\cos(t)\sin(t)\cos(s) dt &= \frac{1}{4} \sin(x)\cos(s)
       \end{aligned}
 \right\}
 \qquad \longrightarrow K_m = \frac{1}{2^{m-1}} \cos(s)\sin(x)
\end{equation}

$\displaystyle\Rightarrow \varphi_m(x) = \frac{\sin(x)}{2^{m-1}}$

\newpage
$$y(x) = 1 + \sin(x)\sum_{m=1}^{\infty}\frac{1}{2^{m-1}} = 2\sin(x) + 1$$

$$y_4(x) = \frac{15 \sin (x)}{8}+1$$

$$\Delta = \frac{\sin(x)}{8}$$

\subsubsection{Резольвента}

$$R(x,s,\lambda) = \cos(s)\sin(x)\sum_{m=1}^{\infty} \frac{1}{2^{m-1}} = 2\cos(s)\sin(x)$$

$$y(x) = 2\sin(x) + 1$$

\subsubsection{Метод вырожденного ядра}

$\displaystyle\alpha(x) = \sin(x),\;\;\beta(s) = \cos(s) \rightarrow a_{11} = \int_{0}^{\frac{\pi}{2}} \sin(s) \cos(s) ds = \frac{1}{2}, \;\; f_1 = \int_{0}^{\frac{1}{2}} \cos(s)ds = 1 \Rightarrow c_i - a_{ij} c_j = f_i \Longleftrightarrow c_1 = 2$

$$y(x) = 2\sin(x) + 1$$

\subsection{$\displaystyle y(x) = \frac{1}{2} \int_{0}^{1} x~ \mathrm{exp}(s) y(s)ds + \mathrm{exp}(-x)$}

$K(x,s) = K_1(x,s) = x\exp(s)$, $f(x)= \varphi_0(x) = \exp(-x)$

\subsubsection{Метод последовательных приближений}

\begin{equation}
 \left.\begin{aligned}
        K_2 = \int_{0}^{1} x\exp(t)t\exp(s) dt  &= x\exp(s)\\
        K_3 = \int_{0}^{1} x\exp(t)t\exp(s) dt &= x\exp(s)
       \end{aligned}
 \right\}
 \qquad \longrightarrow K_m = x\exp(s) \Rightarrow \varphi_m(x) = x
\end{equation}

$$y(x) = \exp(-x) + x\sum_{m=1}^{\infty} \frac{1}{2^m} = \exp(-x) + x$$

$$y_4(x) = \frac{15 x}{16}+\exp(-x)$$

$$\Delta = \frac{x}{16}$$

\newpage
\subsubsection{Резольвента}

$$R\left(x,s,\frac{1}{2}\right) = x\exp(s)\sum_{m=1}^{\infty}\frac{1}{2^{m-1}} = 2x\exp(s)$$

$$y(x) = x + \exp(-x)$$

\subsubsection{Метод вырожденного ядра}

$\displaystyle\alpha(x) = x,\;\;\beta(s) = \exp(s) \rightarrow a_{11} = \int_{0}^{1} s \exp(s) ds = 1, \;\; f_1 = \int_{0}^{1} ds = 1 \Rightarrow c_1 - a_{11} c_1 = f_1 \Longleftrightarrow\\
\Longleftrightarrow c_1 = 2$

$$y(x) = x + \exp(s)$$

\subsection{$\displaystyle y(x) =  \int_{0}^{\frac{1}{2}} y(s)ds + x$}

$K(x,s) = K_1(x,s) = 1$, $f(x)= \varphi_0(x) = x$

\subsubsection{Метод последовательных приближений}

\begin{equation}
 \left.\begin{aligned}
        K_2 = \int_{0}^{\frac{1}{2}} dt  &= \frac{1}{2}\\
        K_3 = \int_{0}^{\frac{1}{2}} \frac{1}{2} dt &= \frac{1}{4}
       \end{aligned}
 \right\}
 \qquad \longrightarrow K_m = \frac{1}{2^{m-1}} \Rightarrow \varphi_m(x) = \frac{1}{2^{m+2}}
\end{equation}

$$y(x) = x + \sum_{m=1}^{\infty}\frac{1}{2^{m+2}} = x + \frac{1}{4}$$

$$y_4(x) = x + \frac{15}{64}$$

$$\Delta = \frac{1}{64}$$

\subsubsection{Резольвента}

$$R(x,s,\lambda) = \sum_{m=1}^{\infty} \frac{1}{2^{m+1}} = \frac{1}{2}$$

$$y(x) = \frac{1}{4} + x$$

\newpage
\subsubsection{Метод вырожденного ядра}

$\displaystyle\alpha(x) = 1,\;\;\beta(s) = 1 \rightarrow a_{11} = \int_{0}^{\frac{1}{2}} ds = \frac{1}{2}, \;\; f_1 = \int_{0}^{\frac{1}{2}} s ds = \frac{1}{8} \Rightarrow c_i - a_{ij} c_j = f_i \Longleftrightarrow c_1 = \frac{1}{4}$

$$y(x) = \frac{1}{4} + x$$

\section{Интегральные уравнения Вольтерра}

\subsection{Задание 1}
Преобразуйте интегральные уравнения Вольтерра первого рода в уравнения Вольтерра второго рода и решите их.
\subsubsection{$\displaystyle \int_{0}^{x}  \mathrm{sin}(x-s) y(s)ds =  \mathrm{exp}\left(\frac{x^2}{2} - 1\right)$}

$$\left(\int_{0}^{x} \sin(x-s) y(s) ds\right)_{xx} = \left(\int_{0}^{x} \cos(s-x) y(s) ds\right)_x = y(x) + \int_{0}^{x} \sin(s-x) y(s) ds$$

$$\left(\exp\left(\frac{x^2}{2}\right)\right)_{xx} = \left(x \exp\left(\frac{x^2}{2}\right)\right)_x = \left(x^2 + 1\right)\exp\left(\frac{x^2}{2}\right)$$

$\displaystyle \Rightarrow y(x) + \int_{0}^{x} \sin(s-x) y(s) ds = \left(x^2 + 1\right)\exp\left(\frac{x^2}{2}\right)$

$\displaystyle\left(y(x) = \left(x^2 + 1\right)\exp\left(\frac{x^2}{2}\right) - \int_{0}^{x} \sin(s-x) y(s) ds\right)_{xx} \rightarrow y''(x) = y(x) + \int_{0}^{x} \sin(s-x) y(s) ds + \left(x^4 + 6 x^2 + 3\right) \exp\left(\frac{x^2}{2}\right) = \left(x^4 + 7x^2 + 4\right)\exp\left(\frac{x^2}{2}\right)$

$$y'(x) = \int_{0}^{x} \left(t^4 + 7t^2 + 4\right)\exp\left(\frac{t^2}{2}\right)dt = (x^2 + 4)x\exp\left(\frac{x^2}{2}\right)$$

$$y(x) = \int_{0}^{x} (x^2 + 4)x\exp\left(\frac{x^2}{2}\right)dx - 1 = \left(x^2 + 2\right)\exp\left(\frac{x^2}{2}\right) - 1$$

\subsubsection{$\displaystyle \int_{0}^{x}  \mathrm{exp}(x-s) y(s)ds =  \mathrm{sin}(x)$}

$$\left(\int_{0}^{x} \exp(x-s) y(s) ds\right)_{x} = y(x) + \int_{0}^{x} \exp(x-s) y(s) ds$$

$$\left( \sin(x) \right)_{x} = \cos(x)$$

$\displaystyle \Rightarrow y(x) = \cos(x) - \int_{0}^{x} \exp(x-s) y(s) ds$

\begin{equation*}
 \left.\begin{aligned}
        K &= K_1 = -\exp(x-s)\\
        K_2 &= \int_{s}^{x} -\exp(x-t)\exp(t-s) dt  = (x-s)\exp(x-s)\\
        K_3 &= \int_{s}^{x} -\exp(x-t)(t-s)\exp(t-s) dt = -\frac{1}{2}(x-s)^2 \exp(x-s)\\
        K_4 &= \int_{s}^{x} \frac{1}{2}\exp(x-t)(t-s)^2 \exp(t-s) dt = \frac{1}{6}(x-s)^4 \exp(x-s)
       \end{aligned}
 \right\}
 \qquad K_m = (-1)^{m}\frac{(x-s)^{m-1}}{(m-1)!}\exp(x-s)
\end{equation*}



$$R(x,s) = \exp(x-s)\sum_{m=1}^{\infty} (-1)^{m}\frac{(x-s)^{m-1}}{(m-1)!} = -1$$

$$y(x) = -\sin(x) + \cos(x)$$

\subsection{Задание 2}
Вычислить резольвенту для ядра интегрального уравнения Вольтерра второго рода, если ядро уравнения $K(x,s) = \exp(x-s)$.

\begin{equation*}
 \left.\begin{aligned}
        K &= K_1 = \exp(x-s)\\
        K_2 &= \int_{s}^{x} \exp(x-t)\exp(t-s) dt  = (x-s)\exp(x-s)\\
        K_3 &= \int_{s}^{x} \exp(x-t)(t-s)\exp(t-s) dt = \frac{1}{2}(x-s)^2 \exp(x-s)\\
        K_4 &= \int_{s}^{x} \frac{1}{2}\exp(x-t)(t-s)^2 \exp(t-s) dt = \frac{1}{6}(x-s)^3 \exp(x-s)
       \end{aligned}
 \right\}
 \qquad K_m = \frac{(x-s)^{m-1}}{(m-1)!}\exp(x-s)
\end{equation*}

$$R(x,s) = \exp(x-s)\sum_{m=1}^{\infty} \frac{(x-s)^{m-1}}{(m-1)!} = \exp(2(x-s))$$

\subsection{Задание 3}
С помощью эквивалентного дифференциального уравнения решить интегральное уравнение Вольтерра второго рода
$$y(x) = \int_{0}^{x} x s y(s) ds + x$$
\begin{enumerate}
  \item $$\left(\int_{0}^{x} xsy(s) ds\right)_{xx} = \left(x^2 y(x) + \int_{0}^{x} sy(s) ds\right)_x = x^2 y'(x) + 3xy(x)$$

$$y''(x) - x^2 y'(x) - 3xy(x) = 0, \;\; y(0) = 0, \;\; y'(0) = 1$$

$$y(x) = x\exp\left(\frac{x^3}{3}\right)$$

  \item
  \fbox{
 \addtolength{\linewidth}{-30\fboxsep}
 \addtolength{\linewidth}{-30\fboxrule}
 \begin{minipage}{\linewidth}
  \begin{equation}
    K(x,s) = \sum_{i=1}^{m} \alpha_i(x) \beta_i(s), \;\; y(x) = \sum_{i=1}^{m} \alpha_i(x) \int_{a}^{x} \beta_i(s) y(s) ds + f(x)
  \end{equation}
  \begin{equation}
    \upsilon'_i(x) = \beta_i(x)\left[\sum_{i=1}^{m} \alpha_i(x) \upsilon_i(x) + f(x)\right],\;\; \upsilon_i(a) = 0
  \end{equation}
  \begin{equation}
    y(x) = \frac{\upsilon'_i(x)}{\beta_i(x)}
  \end{equation}
 \end{minipage}
}
  $\displaystyle\alpha(x) = x,\;\; \beta(s) = s,\;\; f(x) = x$
  $$\upsilon(x) = \int_{0}^{x} s y(s) ds$$
  $$\upsilon'(x) = x[x\upsilon(x) + x] = x^2 \upsilon(x) + x^2, \;\; \upsilon(0) = 0$$
$$\upsilon^o(x) = C\exp\left(\frac{x^3}{3}\right), \;\; C'(x)\exp\left(\frac{x^3}{3}\right) = x^2 \Rightarrow C(x) = K - \exp\left(-\frac{x^3}{3}\right)$$
$$\upsilon(x) = C(x)\exp\left(\frac{x^3}{3}\right) = 1 - \exp\left(\frac{x^3}{3}\right)$$
$$y(x) = \frac{\upsilon(x)}{\beta(x)} = x\exp\left(\frac{x^3}{3}\right)$$
\end{enumerate}

\subsection{Задание 4}
С помощью эквивалентного дифференциального уравнения решить интегральное уравнение Вольтерра второго рода
$$y(x) = \sum_{i=1}^{m} \alpha_i(x) \int_{a}^{x} \beta_i(s) y(s) ds + f(x)$$
при $m=1$ задано $\alpha_1(x) = -\frac{1}{x \cos(x)}$, $\beta_1(s) = \cos(s)$, $f(x) = \frac{x^2}{\cos(x)}$.

$$y(x) = -\frac{1}{x\cos(x)}\int_{a}^{x} \cos(s)y(s)ds + \frac{x^2}{\cos(x)}$$

$$\upsilon'(x) = \cos(x)\left[-\frac{1}{x \cos(x)} \upsilon(x) + \frac{x^2}{\cos(x)}\right] = -\frac{1}{x}\upsilon(x) + x^2, \;\; \upsilon(a) = 0$$

$$\upsilon^o(x) = \frac{C}{x}, \;\; \upsilon^{p}(x) = A x^3 = \frac{1}{4} x^3$$

$$\upsilon(x) = \upsilon^o(x) + \upsilon^{p}(x) = \frac{C}{x} + \frac{x^3}{4} = \frac{x^3}{4} -\frac{a^4}{4x}$$

$$y(x) = \frac{3x^4 + a^4}{4x^2\cos(x)}$$

\subsection{Задание 5}
Решить с помощью резольвенты (разрешающего ядра)
\subsubsection{$\displaystyle y(x) =  \int_{0}^{x} s y(s)ds + 1$}

\begin{equation*}
 \left.\begin{aligned}
        K &= K_1 = s\\
        K_2 &= \int_{s}^{x} ts dt  = \frac{1}{2}s\left(x^2 - s^2\right)\\
        K_3 &= \int_{s}^{x} \frac{1}{2} t s\left(t^2 - s^2\right) dt = \frac{1}{8}s(x^2-s^2)^2\\
        K_4 &= \int_{s}^{x} \frac{1}{8} ts(t^2-s^2)^2 dt = \frac{1}{48}s(x^2-s^2)^3
       \end{aligned}
 \right\}
 \qquad K_m = s\frac{\left(x^2-s^2\right)^{m-1}}{2^{m-1} (m-1)!}
\end{equation*}

$$R(x,s) = s \sum_{m=1}^{\infty} \frac{\left(x^2-s^2\right)^{m-1}}{2^{m-1} (m-1)!} = s\exp\left(\frac{x^2-s^2}{2}\right)$$

$$y(x) = 1 + \exp\left(\frac{x^2}{2}\right)\int_{0}^{x} s\exp\left(-\frac{s^2}{2}\right)ds = \exp\left(\frac{x^2}{2}\right)$$

\subsubsection{$\displaystyle y(x) =  -\int_{0}^{x} (x-s) y(s)ds + x$}

\begin{equation*}
 \left.\begin{aligned}
        K &= K_1 = s-x\\
        K_2 &= \int_{s}^{x} (t-x)(s-t) dt  = -\frac{1}{6}(s-x)^3\\
        K_3 &= \int_{s}^{x} \frac{1}{6} (x-t) (t-s)^3 dt = \frac{1}{120}(s-x)^5\\
        K_4 &= \int_{s}^{x} \frac{1}{120} (x-t) (s-t)^5 = -\frac{1}{5040}(s-x)^7
       \end{aligned}
 \right\}
 \qquad K_m = (-1)^m\frac{(s-x)^{2m-1}}{(2m-1)!}
\end{equation*}

$$R(x,s) = \sum_{m=1}^{\infty} (-1)^m\frac{(s-x)^{2m-1}}{(2m-1)!} = -\sin(s-x)$$

$$y(x) = x + \int_{0}^{x} \sin(s-x) s ds = \sin(x)$$

\subsubsection{$\displaystyle y(x) =  -\int_{0}^{x} y(s)ds + \frac{x^2}{2} + x$}

\begin{equation*}
 \left.\begin{aligned}
        K &= K_1 = -1\\
        K_2 &= \int_{s}^{x} dt  = x-s\\
        K_3 &= \int_{s}^{x} (s-t) dt = -\frac{1}{2}(x-s)^2\\
        K_4 &= \int_{s}^{x} \frac{1}{2}(t-s)^2 = \frac{1}{6}(x-s)^3
       \end{aligned}
 \right\}
 \qquad K_m = (-1)^m\frac{(x-s)^{m-1}}{(m-1)!}
\end{equation*}

$$R(x,s) = \sum_{m=1}^{\infty} (-1)^m\frac{(x-s)^{m-1}}{(m-1)!} = -\exp(s-x)$$

$$y(x) = \frac{x^2}{2} + x - \int_{0}^{x} \exp(s-x)\left(\frac{s^2}{2} + s\right) ds = x$$

\end{document} 