% !TEX encoding = UTF-8 Unicode
\documentclass[
11pt,
master, % тип документа
subf, % подключить и настроить пакет subfig для вложенной нумерации рисунков
href, % подключить и настроить пакет hyperref
colorlinks=true, % цветные гиперссылки
times, % шрифт Times как основной
%fixint=false % отключить прямые знаки интегралов
]{disser}
\usepackage[left=25mm, top=20mm, right=10mm, bottom=20mm]{geometry}
\usepackage[T2A]{fontenc}
\usepackage[utf8]{inputenc}
\usepackage[english,russian]{babel}
\usepackage{amsmath,amssymb,cmap} % cmap для кодировки шрифтов в pdf
\usepackage{pdfpages} % вставляем pdf файлы
\usepackage{indentfirst} % отделять первую строку раздела абзацным отступом
\usepackage{titletoc} % убираем отступ перед "Оглавление"
\usepackage{graphicx}
\usepackage{setspace}
\usepackage{verbatim} % для оформления кода
\usepackage{pdfsync} % установка соответствия документ - код
\graphicspath{{./Img/}}

\setlength\parindent{5ex} % абзацный отступ равный пяти строчным буквам основного шрифта
\pagestyle{plain} % включаем нумерацию
\setcounter{tocdepth}{2} % включать подсекции в оглавление
\linespread{1.3} % полуторный интервал

% Номера страниц снизу и по центру
\pagestyle{footcenter}
\chapterpagestyle{footcenter}

\begin{document}
	
\pagestyle{empty}
\begin{center}
	
	\noindent  Федеральное государственное бюджетное образовательное учреждение\\
	высшего профессионального образования\\
	
	Московский государственный технический университет им. Н.Э. Баумана \\
	Факультет <<Фундаментальные науки>>\bigskip\\
	
	\vfill
	
	Лабораторная работа №3\\
	по курсу «Вычислительная физика»\\
	Тема: «Формулы численного дифференцирования»\\
	\textbf{Вариант 6}\\
	
	
	\vfill
	\vfill
	\begin{flushright}
		\begin{tabular}{ll}
			Выполнили: & студенты группы ФН4-72Б     \\
			& Мистрюкова Л.А., Хижик А.И.  \\
			Проверил:  & доцент, к.физ.-мат.н.       \\
			& Хасаншин Р.Х.
		\end{tabular}
	\end{flushright}
	\vfill
	\begin{center}
		Москва, $2019$
	\end{center}
	
\end{center}
\pagebreak


\pagestyle{plain}

\tableofcontents

\section{Теоретическая часть}
\subsection{Задача численного дифференцирования}
К численному дифференцированию приходится прибегать в случае, когда функция $f(x)$, для которой ищется производная, задана таблично или функциональная зависимость $x$ и $f(x)$ имеет сложное аналитическое выражение.

В этих случаях вместо функции $f(x)$ рассматривают интерполирующую функцию $\varphi(x)$ и считают производную от $f(x)$ приближённо равной производной $\varphi(x)$. Естественно, при этом производная от $f(x)$ будет найдена с некоторой погрешностью.

Функцию $f(x)$ можно записать в следующем виде:
$$f(x) = \varphi(x) + R(x),$$
где $R(x)$ -- остаточный член интерполяционной формулы. Дифференцируя это тождество $k$ раз (предполагая, что $f(x)$ и $\varphi(x)$ имеют производные $k$-го порядка), получим
$$f^{(k)}(x) = \varphi^{(k)}(x) + R^{(k)}(x).$$
Так как за приближённое значение $f^{(k)}(x)$ принимается $\varphi^{(k)}(x)$, то погрешность есть $R^{(k)}(x)$.

\subsection{Формулы численного дифференцирования}
Имеем функцию $f(x)$ непрерывную на $[a,b]$. Введем на $[a,b]$ сетку:
$$x_0<x_1<...<x_n, ~ x_i\in[a,b], ~ i=\overline{0,n}.$$
$$f_i = f(x_i).$$

Если принять, что узлы равноотстоящие и шаг сетки равен $h$, т.е.
\begin{equation}
x_i=x_0 + ih, ~ h>0, ~ i=\overline{0,n},
\end{equation}

то будут верны следующие формулы для расчета первой производной в узлах:
\begin{enumerate}
\item Если $f(x) \in C_2 [x_0, x_1]$, то на отрезке $[x_0,x_1]$ существует такая точка $\xi$, что
\begin{equation}
f_o^{'}=\frac{f_1-f_0}{h}-\frac{h}{2}f^{''}_{\xi}.
\end{equation}

\item Если  $f(x) \in C_3 [x_{-1}, x_1]$, то на $[x_{-1}, x_1]$ существует такая точка $\xi$, что
\begin{equation}
f_o^{'}=\frac{f_{1}-f_{-1}}{2h}-\frac{h^2}{6}f^{(3)}_{\xi}.
\end{equation}


\item Если $f(x) \in C_4 [x_{-1}, x_1]$, то существует такая точка $\xi$, что
\begin{equation}
f_o^{''}=\frac{f_{1}-2f_0+f_{-1}}{h^2}-\frac{h^2}{12}f^{(4)}_{\xi}.
\end{equation}
\end{enumerate}


\subsubsection{Случай интерполяционного многочлена Лагранжа}
Интерполяционный многочлен Лагранжа:
$$L_n(x) = \sum_{k=0}^{n}f_k\frac{\omega_{n+1}(x)}{(x-x_k)\omega_n'(x)} = \sum_{k=0}^{n}f_k\prod_{i\neq k}^{n}\frac{x-x_i}{x_k-x_i}.$$
$$R_n(x) = \frac{f^{(n+1)}(\xi)}{(n+1)!}\omega_{n+1}(x).$$

\begin{spacing}{1.0}

\newpage
\section{Постановка задачи}
Получить аппроксимации производных $m$-го порядка, используя интерполяционные многочлены Лагранжа и Ньютона $n$-й степени.

\begin{itemize}
  \item При $m=1$, $n=3$ в точке $x_0$;
  \item При $m=2$, $n=4$ в точке $x_2$.
\end{itemize}

\newpage
\section{Вычисление производных}

Для $n=3$: $f(x) = L_3(x) + R_3(x)$, где

$\displaystyle L_3(x) = \sum_{k=0}^{3}f_k\prod_{i\neq k}^{3}\frac{x-x_i}{x_k-x_i} =f_0\frac{(x-x_1)(x-x_2)(x-x_3)}{(x_0-x_1)(x_0-x_2)(x_0-x_3)}+\\
+f_1\frac{(x-x_0)(x-x_2)(x-x_3)}{(x_1-x_0)(x_1-x_2)(x_1-x_3)}+f_2\frac{(x-x_0)(x-x_1)(x-x_3)}{(x_2-x_0)(x_2-x_1)(x_2-x_3)}+\\
+f_3\frac{(x-x_0)(x-x_1)(x-x_3)}{(x_3-x_0)(x_3-x_1)(x_3-x_2)}$;\\

$\displaystyle R_3(x) = \frac{f^{(4)}(\xi)}{4!}\omega_{4}(x) = \frac{f^{(4)}(\xi)}{4!}(x-x_0)(x-x_1)(x-x_2)(x-x_3)$.\\

Первая производная $f(x)$:\\
$\displaystyle f'(x) = L'_3(x) + R'_3(x) =\\
=f_0\frac{(x-x_2)(x-x_3)+(x-x_1)(x-x_3)+(x-x_1)(x-x_2)}{(x_0-x_1)(x_0-x_2)(x_0-x_3)}+\\
+f_1\frac{(x-x_2)(x-x_3)+(x-x_0)(x-x_3)+(x-x_0)(x-x_2)}{(x_1-x_0)(x_1-x_2)(x_1-x_3)}+\\
+f_2\frac{(x-x_1)(x-x_3)+(x-x_0)(x-x_3)+(x-x_0)(x-x_1)}{(x_2-x_0)(x_2-x_1)(x_2-x_3)}+\\
+f_3\frac{(x-x_1)(x-x_2)+(x-x_0)(x-x_2)+(x-x_0)(x-x_1)}{(x_3-x_0)(x_3-x_1)(x_3-x_2)}+\\
+\frac{f^{(4)}(\xi)}{4!}((x-x_1)(x-x_2)(x-x_3)+(x-x_0)(x-x_2)(x-x_3)+\\
+(x-x_0)(x-x_1)(x-x_3)+(x-x_0)(x-x_1)(x-x_2))$.\\

Значение в точке $x_0$:\\
$\displaystyle f'(x_0) =\\
=f_0\frac{(x_0-x_2)(x_0-x_3)+(x_0-x_1)(x_0-x_3)+(x_0-x_1)(x_0-x_2)}{(x_0-x_1)(x_0-x_2)(x_0-x_3)}+\\
+f_1\frac{(x_0-x_2)(x_0-x_3)}{(x_1-x_0)(x_1-x_2)(x_1-x_3)}+\\
+f_2\frac{(x_0-x_1)(x_0-x_3)}{(x_2-x_0)(x_2-x_1)(x_2-x_3)}+\\
+f_3\frac{(x_0-x_1)(x_0-x_2)}{(x_3-x_0)(x_3-x_1)(x_3-x_2)}+\\
+\frac{f^{(4)}(\xi)}{4!}(x_0-x_1)(x_0-x_2)(x_0-x_3)$.\\

\end{spacing}

Для равномерной сетки с шагом $h>0$: $x_i = x_0 + ih$\\
$\displaystyle f'(x_0) = \frac{1}{6h}(-11f_0 + 18f_1 - 9f_2 + 2f_3) - \frac{f^{(4)}(\xi)}{4}h^3$.\\

Для $n=4$:\\
$\displaystyle L_4(x) = f_0\frac{(x-x_1)(x-x_2)(x-x_3)(x-x_4)}{(x_0-x_1)(x_0-x_2)(x_0-x_3)(x_0-x_4)}+\\
+f_1\frac{(x-x_0)(x-x_2)(x-x_3)(x-x_4)}{(x_1-x_0)(x_1-x_2)(x_1-x_3)(x_1-x_4)}+\\
+f_2\frac{(x-x_0)(x-x_1)(x-x_3)(x-x_4)}{(x_2-x_0)(x_2-x_1)(x_2-x_3)(x_2-x_4)}+\\
+f_3\frac{(x-x_0)(x-x_1)(x-x_2)(x-x_4)}{(x_3-x_0)(x_3-x_1)(x_3-x_2)(x_3-x_4)}+\\
+f_4\frac{(x-x_0)(x-x_1)(x-x_2)(x-x_3)}{(x_4-x_0)(x_4-x_1)(x_4-x_2)(x_4-x_3)}$;\\

$\displaystyle R_4(x) = \frac{f^{(5)}(\xi)}{5!}(x-x_0)(x-x_1)(x-x_2)(x-x_3)(x-x_4)$.\\

Первая производная $f(x)$:\\
\small
$\displaystyle f'(x) = L'_4(x) + R'_4(x) =\\
=f_0(\frac{(x-x_2)(x-x_3)(x-x_4)+(x-x_1)(x-x_3)(x-x_4)+(x-x_1)(x-x_2)(x-x_4)}{(x_0-x_1)(x_0-x_2)(x_0-x_3)(x_0-x_4)}+\\
+ \frac{(x-x_1)(x-x_2)(x-x_3)}{(x_0-x_1)(x_0-x_2)(x_0-x_3)(x_0-x_4)})
+f_1(\frac{(x-x_2)(x-x_3)(x-x_4)+(x-x_0)(x-x_3)(x-x_4)}{(x_1-x_0)(x_1-x_2)(x_1-x_3)(x_1-x_4)}+ \\
+ \frac{(x-x_0)(x-x_2)(x-x_4)+(x-x_0)(x-x_2)(x-x_3)}{(x_1-x_0)(x_1-x_2)(x_1-x_3)(x_1-x_4)})
+f_2(\frac{(x-x_1)(x-x_3)(x-x_4)}{(x_2-x_0)(x_2-x_1)(x_2-x_3)(x_2-x_4)}+ \\
+ \frac{(x-x_0)(x-x_3)(x-x_4)+(x-x_0)(x-x_1)(x-x_4)+(x-x_0)(x-x_1)(x-x_3)}{(x_2-x_0)(x_2-x_1)(x_2-x_3)(x_2-x_4)})+\\
+f_3(\frac{(x-x_1)(x-x_2)(x-x_4)+(x-x_0)(x-x_2)(x-x_4)+(x-x_0)(x-x_1)(x-x_4)}{(x_3-x_0)(x_3-x_1)(x_3-x_2)(x_3-x_4)} + \\
+ \frac{(x-x_0)(x-x_1)(x-x_2)}{(x_3-x_0)(x_3-x_1)(x_3-x_2)(x_3-x_4)})
+f_4(\frac{(x-x_1)(x-x_2)(x-x_3)+(x-x_0)(x-x_2)(x-x_3)}{(x_4-x_0)(x_4-x_1)(x_4-x_2)(x_4-x_3)} + \\
+ \frac{(x-x_0)(x-x_1)(x-x_3)+(x-x_0)(x-x_1)(x-x_2)}{(x_4-x_0)(x_4-x_1)(x_4-x_2)(x_4-x_3)})
+\frac{f^{(5)}(\xi)}{5!}((x-x_1)(x-x_2)(x-x_3)(x-x_4)+(x-x_0)(x-x_2)(x-x_3)(x-x_4)+(x-x_0)(x-x_1)(x-x_3)(x-x_4)+(x-x_0)(x-x_1)(x-x_2)(x-x_4)+(x-x_0)(x-x_1)(x-x_2)(x-x_3))$.\\

\normalsize
Вторая производная $f(x)$:\\
\small
$\displaystyle f''(x) = L''_4(x) + R''_4(x) =\\
=f_0(\frac{(x-x_3)(x-x_4)+(x-x_2)(x-x_4)+(x-x_2)(x-x_3)+(x-x_3)(x-x_4)+(x-x_1)(x-x_3)}{(x_0-x_1)(x_0-x_2)(x_0-x_3)(x_0-x_4)}+ \\
+ \frac{(x-x_1)(x-x_4) + (x-x_2)(x-x_4)+(x-x_1)(x-x_4)+(x-x_1)(x-x_2)+(x-x_2)(x-x_3)}{(x_0-x_1)(x_0-x_2)(x_0-x_3)(x_0-x_4)}+\\
+ \frac{(x-x_1)(x-x_3)+(x-x_1)(x-x_2)}{(x_0-x_1)(x_0-x_2)(x_0-x_3)(x_0-x_4)})
+f_1(\frac{(x-x_3)(x-x_4)+(x-x_2)(x-x_4)+(x-x_2)(x-x_3)}{(x_1-x_0)(x_1-x_2)(x_1-x_3)(x_1-x_4)} + \\
+ \frac{(x-x_3)(x-x_4)+(x-x_0)(x-x_4)+(x-x_0)(x-x_3)+(x-x_2)(x-x_4)+(x-x_0)(x-x_4)}{(x_1-x_0)(x_1-x_2)(x_1-x_3)(x_1-x_4)}+\\
+ \frac{(x-x_0)(x-x_2)+(x-x_2)(x-x_3)+(x-x_0)(x-x_3)+(x-x_0)(x-x_2)}{(x_1-x_0)(x_1-x_2)(x_1-x_3)(x_1-x_4)})+\\
+f_2(\frac{(x-x_3)(x-x_4)+(x-x_1)(x-x_4)+(x-x_1)(x-x_3)+(x-x_3)(x-x_4)+(x-x_0)(x-x_4)}{(x_2-x_0)(x_2-x_1)(x_2-x_3)(x_2-x_4)}+ \\
+ \frac{(x-x_0)(x-x_3)+(x-x_1)(x-x_4)+(x-x_0)(x-x_4)+(x-x_0)(x-x_1)+(x-x_1)(x-x_3)}{(x_2-x_0)(x_2-x_1)(x_2-x_3)(x_2-x_4)}+\\
+ \frac{(x-x_0)(x-x_3)+(x-x_0)(x-x_1)}{(x_2-x_0)(x_2-x_1)(x_2-x_3)(x_2-x_4)})
+f_3(\frac{(x-x_2)(x-x_4)+(x-x_1)(x-x_4)+(x-x_1)(x-x_2)}{(x_3-x_0)(x_3-x_1)(x_3-x_2)(x_3-x_4)}+ \\
+ \frac{(x-x_2)(x-x_4)+(x-x_0)(x-x_4)+(x-x_0)(x-x_2)+(x-x_1)(x-x_4)+(x-x_0)(x-x_4)}{(x_3-x_0)(x_3-x_1)(x_3-x_2)(x_3-x_4)}+\\
+ \frac{(x-x_0)(x-x_1)+(x-x_1)(x-x_2)+(x-x_0)(x-x_2)+(x-x_0)(x-x_1)}{(x_3-x_0)(x_3-x_1)(x_3-x_2)(x_3-x_4)})+\\
+f_4(\frac{(x-x_2)(x-x_3)+(x-x_1)(x-x_3)+(x-x_1)(x-x_2)+(x-x_2)(x-x_3)+(x-x_0)(x-x_3)}{(x_4-x_0)(x_4-x_1)(x_4-x_2)(x_4-x_3)}+ \\
+ \frac{(x-x_0)(x-x_2)+(x-x_1)(x-x_3)+(x-x_0)(x-x_3)+(x-x_0)(x-x_1)+(x-x_1)(x-x_2)}{(x_4-x_0)(x_4-x_1)(x_4-x_2)(x_4-x_3)} + \\
+ \frac{(x-x_0)(x-x_2)+(x-x_0)(x-x_1)}{(x_4-x_0)(x_4-x_1)(x_4-x_2)(x_4-x_3)})
+\frac{f^{(5)}(\xi)}{5!}((x-x_2)(x-x_3)(x-x_4)+(x-x_1)(x-x_3)(x-x_4)+(x-x_1)(x-x_2)(x-x_4)+(x-x_1)(x-x_2)(x-x_3)+(x-x_2)(x-x_3)(x-x_4)+(x-x_0)(x-x_3)(x-x_4)+(x-x_0)(x-x_2)(x-x_4)+(x-x_0)(x-x_2)(x-x_3)+(x-x_1)(x-x_3)(x-x_4)+(x-x_0)(x-x_3)(x-x_4)+(x-x_0)(x-x_1)(x-x_4)+(x-x_0)(x-x_1)(x-x_3)+(x-x_1)(x-x_2)(x-x_4)+(x-x_0)(x-x_2)(x-x_4)+(x-x_0)(x-x_1)(x-x_4)+(x-x_0)(x-x_1)(x-x_2)+(x-x_1)(x-x_2)(x-x_3)+(x-x_0)(x-x_2)(x-x_3)+(x-x_0)(x-x_1)(x-x_3)+(x-x_0)(x-x_1)(x-x_2))$.\\
\\


\normalsize
Значение в точке $x_2$:\\
\small
$\displaystyle f''(x_2) =\\
=f_0(\frac{(x_2-x_3)(x_2-x_4)+(x_2-x_3)(x_2-x_4)+(x_2-x_1)(x_2-x_3)+(x_2-x_1)(x_2-x_4)}{(x_0-x_1)(x_0-x_2)(x_0-x_3)(x_0-x_4)}+\\
+\frac{(x_2-x_1)(x_2-x_4)+(x_2-x_1)(x_2-x_3)}{(x_0-x_1)(x_0-x_2)(x_0-x_3)(x_0-x_4)})+\\
+f_1(\frac{(x_2-x_3)(x_2-x_4)+(x_2-x_3)(x_2-x_4)+(x_2-x_0)(x_2-x_4)+(x_2-x_0)(x_2-x_3)}{(x_1-x_0)(x_1-x_2)(x_1-x_3)(x_1-x_4)}+\\
+\frac{(x_2-x_0)(x_2-x_4)+(x_2-x_0)(x_2-x_3)}{(x_1-x_0)(x_1-x_2)(x_1-x_3)(x_1-x_4)})+\\
+f_2(\frac{(x_2-x_3)(x_2-x_4)+(x_2-x_1)(x_2-x_4)+(x_2-x_1)(x_2-x_3)+(x_2-x_3)(x_2-x_4)}{(x_2-x_0)(x_2-x_1)(x_2-x_3)(x_2-x_4)}+\\
+\frac{(x_2-x_0)(x_2-x_4)+(x_2-x_0)(x_2-x_3)+(x_2-x_1)(x_2-x_4)+(x_2-x_0)(x_2-x_4)}{(x_2-x_0)(x_2-x_1)(x_2-x_3)(x_2-x_4)}+\\
+\frac{(x_2-x_0)(x_2-x_1)+(x_2-x_1)(x_2-x_3)+(x_2-x_0)(x_2-x_3)+(x_2-x_0)(x_2-x_1)}{(x_2-x_0)(x_2-x_1)(x_2-x_3)(x_2-x_4)})+\\
+f_3(\frac{(x_2-x_1)(x_2-x_4)+(x_2-x_0)(x_2-x_4)+(x_2-x_1)(x_2-x_4)+(x_2-x_0)(x_2-x_4)}{(x_3-x_0)(x_3-x_1)(x_3-x_2)(x_3-x_4)}+\\
+\frac{(x_2-x_0)(x_2-x_1)+(x_2-x_0)(x_2-x_1)}{(x_3-x_0)(x_3-x_1)(x_3-x_2)(x_3-x_4)})+\\
+f_4(\frac{(x_2-x_1)(x_2-x_3)+(x_2-x_0)(x_2-x_3)+(x_2-x_1)(x_2-x_3)+(x_2-x_0)(x_2-x_3)}{(x_4-x_0)(x_4-x_1)(x_4-x_2)(x_4-x_3)}+\\
\frac{(x_2-x_0)(x_2-x_1)+(x_2-x_0)(x_2-x_1)}{(x_4-x_0)(x_4-x_1)(x_4-x_2)(x_4-x_3)})+\\
+\frac{f^{(5)}(\xi)}{5!}((x_2-x_1)(x_2-x_3)(x_2-x_4)+(x_2-x_1)(x_2-x_2)(x_2-x_3)+(x_2-x_0)(x_2-x_3)(x_2-x_4)+(x_2-x_1)(x_2-x_3)(x_2-x_4)+(x_2-x_0)(x_2-x_3)(x_2-x_4)+(x_2-x_0)(x_2-x_1)(x_2-x_4)+(x_2-x_0)(x_2-x_1)(x_2-x_3)+(x_2-x_0)(x_2-x_1)(x_2-x_4)+(x_2-x_0)(x_2-x_1)(x_2-x_3))$.\\

\normalsize
Для равномерной сетки с шагом $h>0$: $x_i = x_0 + ih$,\\
$\displaystyle f''(x_2) = \frac{1}{12h^2}[-f_0+16f_1-30f_2+16f_3-f_4].$

\newpage
\subsubsection{Случай интерполяционного многочлена Ньютона}
Интерполяционный многочлен Ньютона:
$$ P_n(x) = f_0 + \sum_{k=1}^{n}f(x_0;\ldots;x_k)\omega_k(x)$$

Для $n=3$: $f(x) = P_3(x) + R_3(x)$, где

$\displaystyle P_3(x) = f_0 + \sum_{k=1}^{3}f(x_0;\ldots;x_k)\omega_k(x) = f_0+(x-x_0)f(x_0;x_1)+\\
+(x-x_0)(x-x_1)f(x_0;x_1;x_2)+(x-x_0)(x-x_1)(x-x_2)f(x_0;x_1;x_2;x_3).$

$\displaystyle R_3(x) = \frac{f^{(4)}(\xi)}{4!}\omega_{4}(x)$.\\

Первая производная $f(x)$:\\
$\displaystyle f'(x) = P'_3(x) + R'_3(x) = f(x_0;x_1)+((x-x_0)+(x-x_1))f(x_0;x_1;x_2)+\\
+((x-x_0)(x-x_1)+(x-x_0)(x-x_2)+(x-x_1)(x-x_2))f(x_0;x_1;x_2;x_3)+\\
+\frac{f^{(4)}(\xi)}{4!}((x-x_1)(x-x_2)(x-x_3)+(x-x_0)(x-x_2)(x-x_3)+\\
+(x-x_0)(x-x_1)(x-x_3)+(x-x_0)(x-x_1)(x-x_2))$.\\

Значение в точке $x_0$:\\
$\displaystyle f'(x_0) = f(x_0;x_1)+(x_0-x_1)f(x_0;x_1;x_2)+(x_0-x_1)(x_0-x_2)f(x_0;x_1;x_2;x_3)+\\
+\frac{f^{(4)}(\xi)}{4!}(x_0-x_1)(x_0-x_2)(x_0-x_3)$.\\

Для равномерной сетки с шагом $h>0$: $x_i = x_0 + ih$,\\
$\displaystyle f'(x_0) = f(x_0;x_1)-hf(x_0;x_1;x_2)+2h^2f(x_0;x_1;x_2;x_3)-\frac{f^{(4)}(\xi)}{4}h^3$

Для $n=4$:\\
$\displaystyle P_4(x) = f_0+(x-x_0)f(x_0;x_1)+(x-x_0)(x-x_1)f(x_0;x_1;x_2)+\\
+(x-x_0)(x-x_1)(x-x_2)f(x_0;x_1;x_2;x_3)+(x-x_0)(x-x_1)(x-x_2)(x-x_3)\times \\
\times f(x_0;x_1;x_2;x_3;x_4)$.\\
$\displaystyle R_4(x) = \frac{f^{(5)}(\xi)}{5!}(x-x_0)(x-x_1)(x-x_2)(x-x_3)(x-x_4)$.\\

Вторая производная $f(x)$:\\
$\displaystyle f''(x) = f(x_0;x_1)+((x-x_0)+(x-x_1))f(x_0;x_1;x_2)+\\
+((x-x_0)(x-x_1)+(x-x_0)(x-x_2)+(x-x_1)(x-x_2))f(x_0;x_1;x_2;x_3)+\\
+((x-x_1)(x-x_2)(x-x_3)+(x-x_0)(x-x_2)(x-x_3)+\\
+(x-x_0)(x-x_1)(x-x_3)+(x-x_0)(x-x_1)(x-x_2))f(x_0;x_1;x_2;x_3;x_4)+\\
+\frac{f^{(5)}(\xi)}{5!}((x-x_2)(x-x_3)(x-x_4)+(x-x_1)(x-x_3)(x-x_4)+(x-x_1)(x-x_2)(x-x_4)+(x-x_1)(x-x_2)(x-x_3)+(x-x_2)(x-x_3)(x-x_4)+(x-x_0)(x-x_3)(x-x_4)+(x-x_0)(x-x_2)(x-x_4)+(x-x_0)(x-x_2)(x-x_3)+(x-x_1)(x-x_3)(x-x_4)+(x-x_0)(x-x_3)(x-x_4)+(x-x_0)(x-x_1)(x-x_4)+(x-x_0)(x-x_1)(x-x_3)+(x-x_1)(x-x_2)(x-x_4)+(x-x_0)(x-x_2)(x-x_4)+(x-x_0)(x-x_1)(x-x_4)+(x-x_0)(x-x_1)(x-x_2)+(x-x_1)(x-x_2)(x-x_3)+(x-x_0)(x-x_2)(x-x_3)+(x-x_0)(x-x_1)(x-x_3)+(x-x_0)(x-x_1)(x-x_2))$.\\

Значение в точке $x_2$:\\
$\displaystyle f''(x_2) = f(x_0;x_1)+((x_2-x_0)+(x_2-x_1))f(x_0;x_1;x_2)+\\
+(x_2-x_0)(x_2-x_1)f(x_0;x_1;x_2;x_3)+\\
+(x_2-x_0)(x_2-x_1)(x_2-x_3)f(x_0;x_1;x_2;x_3;x_4)+\\
+\frac{f^{(5)}(\xi)}{5!}((x_2-x_1)(x_2-x_3)(x_2-x_4)+(x_2-x_0)(x_2-x_3)(x_2-x_4)+(x_2-x_1)(x_2-x_3)(x_2-x_4)+(x_2-x_0)(x_2-x_3)(x_2-x_4)+(x_2-x_0)(x_2-x_1)(x_2-x_4)+(x_2-x_0)(x_2-x_1)(x_2-x_3)+(x_2-x_0)(x_2-x_1)(x_2-x_4)+(x_2-x_0)(x_2-x_1)(x_2-x_3))$.\\

Для равномерной сетки с шагом $h>0$: $x_i = x_0 + ih$,\\
$\displaystyle f''(x_2) = f(x_0;x_1)+3hf(x_0;x_1;x_2)+2h^2f(x_0;x_1;x_2;x_3)-2h^3f(x_0;x_1;x_2;x_3;x_4).$

\section{Вывод}

В данной работе были получены  формулы численного дифференцирования первого и второго порядков для равномерной сетки с шагом $h>0$: $x_i = x_0 + ih$ в случае интерполяционных многочленов Лагранжа и Ньютона:

$$ f'(x_0) = \frac{1}{6h}[-11f_0 + 18f_1 - 9f_2 + 2f_3] - \frac{f^{(4)}(\xi)}{4}h^3;$$
$$ f''(x_2) = \frac{1}{12h^2}[-f_0+16f_1-30f_2+16f_3-f_4];$$

$$ f'(x_0) = f(x_0;x_1)-hf(x_0;x_1;x_2)+2h^2f(x_0;x_1;x_2;x_3)-\frac{f^{(4)}(\xi)}{4}h^3;$$
$$ f''(x_2) =f(x_0;x_1)+3hf(x_0;x_1;x_2)+2h^2f(x_0;x_1;x_2;x_3)-2h^3f(x_0;x_1;x_2;x_3;x_4).$$

\end{document} 